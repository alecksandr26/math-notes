
\section{\S Are Mathematics the Language of Nature?}
Before delving into the concepts of equations, variables, constants, and the accompanying symbolism, it is important to reflect on why mathematics is likely the language of nature. As mentioned in the previous chapter, you cannot physically point to a number, yet we acknowledge their existence because they manifest in various forms around us like they were ghosts. We encounter numbers not only in mathematical textbooks but also in everyday in life. For instance, if you look at the floor where you are standing, you can begin counting the tiles: $1, 2, 3, 4, 5,$ and so on. Similarly, if you examine an apples basket, you can count the apples: $1, 2, 3, 4, 5,$ and so on and so forth. This suggests that numbers are inherently linked to natural patterns. It's like mathematics is the language of nature. I recommend viewing the movie \textbf{\textit{Pi}} if you're keen on exploring this topic further
\footnote{The movie \textit{Pi} (stylized as $\pi$)is a 1998 American conceptual psychological thriller film written and directed by Darren Aronofsky}.

As previously mentioned, Pythagoras stated that the essence, or arche, of the universe is number, and there is little doubt about this. Wherever we look, we see these numerical patterns and ratios. However, through these notes, my intention is to debate this argument. We will find that sometimes nature presents us with incomprehensible and inconsistent patterns, pushing our minds to work at full speed. Nevertheless, it is intriguing to question whether mathematics is the language of nature. What we can be certain of is that we use mathematics to provide the most objective descriptions possible. Returning to the example of the apple, we can give an entirely objective description, which is highly valuable. Consider this: if we had to define everything using human language, which can vary and is subject to the subjectivity of sensations, we would struggle to understand each other\footnote{I recommend reading another article to understand the significance of mathematics as a language. By Krish. called \textit{Mathematics is the language of nature. Random Walk Around The Universe.} \href{https://medium.com/deciphering-the-future/mathematics-is-the-language-of-nature-11a723b21b17
}{link}}. Therefore, whether or not mathematics is the language of nature, I am confident that it is our human way of projecting ourselves beyond our own context—understood through pronouns (you, he, she, I)—to describe the entities or phenomena that surround us\footnote{For a deeper perspective on this matter, I also recommend reading this article by Murphy, P. A. called \textit{Who Says Nature is Mathematical?} \href{https://www.cantorsparadise.com/who-says-nature-is-mathematical-1abdc1330224}{link}}.

Whether mathematics be the language of nature or not, it is crucial to consider it as language. Now, I want to start with symbolism. One of the most important things to understand in mathematics is that when you see the symbol $5$ or the word \textbf{\textit{five}} on a chalkboard or a piece of paper, these are merely representations of that specific number, not the number itself. This concept is analogous to seeing a heart symbol ($\heartsuit$); the symbol is not love itself, but just a representation of that concept. Recognizing this distinction is the first step in understanding math.

Learning mathematics is similar to learning a language, complete with its unique vocabulary and symbols. When you learn vocabulary in a language other than your native language, you initially focus on the letters and their written form. However, over time, to achieve solid comprehension, you begin to associate these words with ideas and concepts, much like pictures in your mind. The same cognitive shift is necessary to grasp the essence of mathematics.

The central concept here pertains to mathematical expressions such as $3 + 2$ or $2 - 1$, aren't just about performing simple arithmetic operations. They actually stand for numbers themselves – for example, $3 + 2$ is equal to $5$, not just because it's the resultant evaluation of the arithmetic operation of the sum $3 + 2$, but also because $3 + 2$ as an expression represents the concept of \textbf{\textit{five}}. This applies to other math expressions too. Sometimes, these expressions don't represent a specific number; instead, they explore the behavior of certain types of numbers in a more general form, and these expressions contain letters (variables and constants), like $x, y, a, b, c,$ instead of numbers.


\section{\S Intro to Algebra}
I intend to begin with a brief overview of algebra, which is widely recognized as one of the oldest branches of mathematics. The earliest records of algebra are found in Babylonian mathematics. In its historical context, algebra was often regarded as one of the foundational disciplines, primarily due to its comprehensive examination of numbers in their most abstract and general form. Although in the contemporary era, some may argue that logic or set theory holds the title of the most fundamental branch of mathematics, it is pertinent to note that algebra continues to maintain its significance. It doesn't really matter what branch of math you study; you are going to encounter algebraic notations such as equations, variables, constants, etc. For that reason, I consider it important to understand at first these fundamental concepts of algebra.

Another reason of why I want to begin with algebra is that much of the mathematics we learn in school is related to algebra in some form. Often, schools attempt to teach these concepts, but they don't allocate enough time for students to truly understand what they mean. Instead, the focus is primarily on the technical processes of solve problems mechanically. Therefore, to genuinely grasp the upcoming notes, we need to study these algebraic concepts in some depth.

As previously mentioned, the origins of mathematics can be traced back to the Sumerians and Babylonians. Specifically, there are records of algebraic expressions among the Babylonians. However, it was not until the contributions of the Greeks, with \textbf{\textit{Diophantus}} (200 to 284 AD), and the Indians, with \textbf{\textit{Brahmagupta}} (598 to 668 AD) and \textbf{\textit{Bhaskara II}} (1114 to 1185 AD), that historians recognize the foundational figures of algebra. Nevertheless, it was with the Arabs in Baghdad, particularly \textit{\textbf{Al-Khwarizmi}} (780 to 850 AD), that algebra became more fully developed and formalized\footnote{I recommend watching the YouTube video \textit{Orígenes del álgebra} to gain a deeper understanding of this topic \href{https://www.youtube.com/watch?v=akS2vnuAol4}{link}}.

The term \textbf{\textit{algebra}} is derived from the Arabic word \textbf{\textit{al-jabr}}, meaning the reunion of broken parts. This term can be traced back to a seminal text authored by the distinguished Arabic mathematician and scientist \textbf{\textit{al-Khwarizmi}}.

The significance of the mathematician \textbf{\textit{al-Khwarizmi}} lies in his authorship of the book \textbf{\textit{al-Jabr}}. This seminal work was the first text to present algebra in an elementary form for its own sake, rather than merely as a tool for solving geometric equations.

The book offers a systematic approach to solving linear and quadratic equations through methods of reduction and completion, or balancing. The initial step involves expressing equations in one of six canonical forms, which constitutes the reduction step. Subsequently, the completion or balancing step removes extraneous terms. Once an equation is in one of these six forms, the methods outlined in the book can be followed to solve for the unknown value $x$\footnote{For more brief history of algebra I recommend to read this article by \textit{Shrouded Science} called \textit{The history of algebra (it’s complicated)} \href{https://shroudedscience.medium.com/the-history-of-algebra-its-complicated-5f806c60de9d}{link}}.

\section{\S Variables, Constants and Expressions}
A \textit{variable} in algebra is one of the most fundamental concepts. Essentially, a variable represents a quantity that can change or vary, depending on certain conditions or information provided. Variables are symbols, often letters: $x, y, z$, ... , that stand for values that can change within the context of a mathematical problem or experiment. In different branches of mathematics, the role and interpretation of a variable can vary.

As previously mentioned, algebra is the study of numbers in their most general form. Consequently, it is common to encounter statements in mathematics containing the words \textbf{\textit{some}} and \textbf{\textit{any}}. These words have their proper mathematical symbolism. However, before exploring that notation, let's start with the historical context. The earliest records of the conceptual use of these words can be found in Greek geometry. The Greeks proved certain propositions in mathematics using the concepts of generalizing \textbf{\textit{any}} and \textbf{\textit{some}} things.

For example, it is evident that with a basic understanding of arithmetic, these equations are indeed true:
\begin{equation} \label{eq:1}
  3 + 1 = 1 + 3,\quad 4 + 1 = 1 + 4,\quad 3 + 2 = 2 + 3
\end{equation}
However, we can generalize these equations into one, because attempting to demonstrate this pattern with all possible numbers would be impractical and imposible. Instead, we can use the power of variables.
\[
  \text{If } x \text{ and } y \text{ stand \textbf{\textit{any}} number two numbers, then }
\]
\[
  x + y = y + x
\]

As you can see, we are condensing all the previous (\ref{eq:1}) equations into a single equation, representing a number with a simple letter. The specific value of the number is not important; what truly matters is demonstrating the general behavior of this kind of numbers. Specifically, we are illustrating one of the most fundamental properties of numbers: the commutative property of addition. Another example:
\[
  \text{If } x \text{ be \textbf{\textit{any}} number there exist \textbf{\textit{some}} numbers } y, \text{ such that }
\]
\[
  y > x
\]

As you can see variables are necessary to the study of numbers into their general form, now the variables
could play differnt roles depending on the expression and its porpuse, for example we could stand the question:
\begin{equation} \label{eq:2}
    \text{For what \textbf{\textit{number or numbers}}}\quad x,
\end{equation}
\[
  x + 2 = 1
\]
  
Expressions of this type are commonly encountered by students in middle school or high school in their math classes. In this context, $x$ is not a known number, and we do not even know the type of number $x$ represents in (\ref{eq:2}). Furthermore, we are not generalizing; rather, $x$ is considered an \textbf{\textit{unknown variable}}. Such expressions might not even exist. For instance, consider the following expression:

\begin{equation} \label{eq:3}
    \text{For what \textbf{\textit{number or numbers}}}\quad x,
\end{equation}
\[
  x + 1 = x + 2
\]
Analyze this equation (\ref{eq:3}). In this instance, there does not exist \textbf{\textit{any}} number such that adding $1$ or $2$ results in the same quantity, which makes false the equation (\ref{eq:3}).

Moving back to the equation (\ref{eq:2}), thus, depending on the equations, when we encounter an \textbf{\textit{unknown variable}}, there could be various possibilities: \textbf{\textit{any}}, \textbf{\textit{some}}, \textbf{\textit{one}}, or \textbf{\textit{none}} values for $x$ that satisfy the equation. For instance, in the case of the equation denoted as (\ref{eq:2}), simple calculations reveal that $x = -1$, indicating that there is only \textbf{\textit{one}} solution. This illustrates one of the fundamental properties of algebraic variables \footnote{Please refer to the chapter titled "Variables" in the book "An Introduction to Mathematics" by A. N. Whitehead for further illustrations and examples.}. Equations are not always true for just one specific value; they can encompass all possible values of the existing types of numbers. As an example:

\begin{equation} \label{eq:4}
    \text{For what \textbf{\textit{number or numbers}}}\quad x,
\end{equation}
\[
  x = x
\]
In this particular equation, we could consider all possible numbers as potential solutions. We seek to determine for which number or numbers this equation holds true.

Returning to equation (\ref{eq:2}). In this scenario, we can accurately state that the solution of equation is a \textbf{\textit{single}} number, $-1$, rather than multiple numbers. Therefore, it becomes evident that \textbf{\textit{some}} lies between \textbf{\textit{one}} and \textbf{\textit{any}}.

Once we comprehend the concepts of \textbf{\textit{some}} and \textbf{\textit{any}}, we can represent them using proper mathematical notation. Typically, the symbol $\exists$ denotes \textbf{\textit{some}} or \textit{exist}, while $\forall$ represents \textbf{\textit{any}} or \textit{for all}.
  
Utilizing the previous examples, we can convey the same ideas using this symbolism. Here, we employ the symbol
$|$ to denote \textbf{\textit{such that}}.
\[
  \exists x \text{ a number } \quad | \quad  x + 2 = 1
\]
Another example:
\[
  \forall x \text{ numbers }, \exists y \text{ a number } \quad | \quad y > x
\]
Now moving to the next concept, equations don't solely involve \textit{unknown variables}; they also incorporate \textbf{\textit{constants}}. However, \textbf{\textit{constants}} are not always specific numerical values like $1, 2, 3, 4, 5$, etc. They can also be represented by letters, such as $a, b, c$, etc.
This practice is commonly employed when aiming to generalize expressions. Although it may initially seem contradictory to novice readers, who associate the term \textbf{\textit{constants}} with something specific, in mathematics—especially in algebra—generalizing concepts involves regarding the components of the expression, such as \textbf{\textit{constants}}, as unspecified in terms of their numerical value, yet known in terms of their type and properties. Notice that with this approach of \textbf{\textit{constants}}, it differs from the concept of an \textit{\textbf{unknown variable}}. For instance, notice that, we can generalize the expression from (\ref{eq:2}) as follows:
\begin{equation} \label{eq:5}
  x + 2 = 1,\quad x + 2 + (-1) = 1 + (-1),\quad x + (2 - 1) = 0, \quad x + b = 0
\end{equation}
By considering and denoting the expression $(2 - 1)$ as $b$, we arrive at a more comprehensive expression. It's important to note that $b$ is a constant, representing a specific type of number, although the exact numerical value remains unknown. At this stage, we haven't yet classified the types of numbers, but this will become clearer in subsequent discussions. Essentially, when utilizing constants like $b$, our objective is to generalize the expression, rather than focusing on individual and specific cases. The primary reason for this is to attain new knowledge. Classifying generality serves as the initial step in constructing knowledge.

Expanding on the notion of generalization, we may consider expressions such as:
\[
  \frac{a}{b},\quad a^{-n},\quad c(a + b), \quad m \cdot x + b
\]
In these cases, except for the last one, we are dealing with purely \textbf{\textit{constants}} expressions. While we aren't specifying particular values, we can confidently identify the types of numbers being utilized. Hence, we continue to refer to them as \textbf{\textit{constants}}.
And finally, there isn't an established standard or rule regarding which letters should be used to represent constants. Thus, we might occasionally use letters like $x$ or $y$ as constants as the beging of this section, although the convention among many mathematicians is to employ the letters $a$, $b$, and so forth.

The final clarification I wish to provide pertains to the concept of \textbf{\textit{expressions}}. As discussed earlier in this chapter, mathematical expressions extend beyond mere collections of arithmetic operations aimed at producing a singular value; they represent diverse methods of describing mathematical entities, particularly numbers. Consider the following two expressions:
\[
  a + b,\quad b + a
\]
These expressions are different, but they describe the same object from various perspectives. Ultimately, they convey the same underlying concept. A clearer example of this concept can be seen in the following expressions.
\begin{equation} \label{eq:6}
  10,\quad 5 + 5,\quad 2 + 3 + 3 + 2,\quad 2 + 3 + 2 + 3,\quad 1 + 1 + 1 + 1 + 1 + 1 + 1 + 1 + 1 + 1
\end{equation}
Notice I'm emphasizing that $10$ qualifies as an expression, serving to represent the concept of \textit{ten}. Similarly, the other expressions listed above function as alternative means of describing the concept of \textit{ten}. Although they may vary a lot, but they ultimately convey the same concept of \textit{ten} from different perspectives.

As demonstrated in equation (\ref{eq:5}), I utilized \textit{parentheses}. \textbf{\textit{Parentheses}} have a specific meaning within the context of arithmetic operations or computational steps. However, in algebra, speaking in terms of generality, they serve to dissect and analyze expressions by delimiting certain parts. This possibility of dissecting and delimiting a part or component of an expression depends on the structure of the expression. For example consider the next expressions:
\[
  x + y + z + u,\quad (x + y) + z + u,\quad x + (y + z) + u,\quad x + (y + z + u)
\]
Here, we are demonstrating another fundamental property of addition, namely the associative property. Although these expressions differ in their structure, they ultimately describe the same underlying concept. As a final example of expressions, consider the various \textit{\textbf{expressions}} of the concept of \textit{$\pi$}.
\begin{equation} \label{eq:12}
  \pi,\quad \frac{4}{1} + ( - \frac{4}{3}) + \frac{4}{5} + (- \frac{4}{7}) + \cdots,\quad 3 + \frac{4}{2 \cdot 3 \cdot 4} + (- \frac{4}{4 \cdot 5 \cdot 6}) + \cdots
\end{equation}
The $\cdots$ indicates that the series of additions of fractions continues. It does not necessarily extend infinitely, but it has the potential to do so.\footnote{A simple infinite series for \textit{$\pi$} is the Gregory–Leibniz series}.

\section{\S Equations, Inequalities and Identities}
Continuing with the basic and fundamental ideas of algebra, we will now discuss the concept of \textbf{\textit{equations}}. In the previous sections, I used \textit{equations} to explain the notions of \textit{variables} and \textit{constants}. You may have noticed that I referred to \textit{expressions} rather than \textit{equations}. This is because \textbf{\textit{equations}} ($=$)\footnote{I recommend watching the video titled \textit{Where do math symbols come from?} by \textit{John David Walters} to understand the origin of the equals ($=$) sign.} are, in essence, mathematical \textbf{\textit{expressions}} or \textit{statements}. As previously mentioned, \textit{expressions} in mathematics describe objects, not necessarily numbers. In the examples down below  (\ref{eq:7}), these types of \textbf{\textit{expressions}}, they express a properties of a mathematical object, in this case the number \textit{ten}.
\begin{equation} \label{eq:7}
  10 = 5 + 5,\quad 5 + 5 = 2 + 3 + 3 + 2,\quad 2 + 3 + 3 + 2 = 10
\end{equation}

Consider these expressions as different facets of the concept of $10$, representing various properties of $10$. While these examples might initially seem trivial, analyzing them in their different descriptions can reveal new properties and qualities of the underlying mathematical object. Consider the next example.
\begin{equation} \label{eq:8}
  10 = 5 + 5,\quad \overset{2 \text{ times } 5}{\overbrace{5 + 5}} = 2 \cdot 5,\quad 10 = 2 \cdot 5,\quad
  \text{ \textit{ten} is an even number}
\end{equation}
Moving towards a more complete description of an \textbf{\textit{equation}}, it can be clearly divided into two parts: two expressions. Consider the following equation.
\begin{equation} \label{eq:9}
  \text{For what $x$ number or numbers}\quad |\quad 2 \cdot x + 10 = 16 + (2 + 2)
\end{equation}
In this equation, we can observe its two components: the \textit{left-hand side}, $2 \cdot x + 10$, and the \textit{right-hand side}, $16 + (2 + 2)$. In this example, we have a variable, $x$, and we might want to determine for which value of $x$ the \textit{expression} $2 \cdot x + 10$ equals $16 + (2 + 2)$. This question about the \textit{unknown variable} leads us to analyze this \textbf{\textit{equation}}, In the sense of determining whether the equation is true or false, we won't know until we examine it. Note that modifying any part of the two \textit{expressions} in the \textbf{\textit{equation}}, either the \textit{left-hand side} or the \textit{right-hand side}, could alter the mathematical object being described. An analogy would be a poem: if we change the sentences, we might change the meaning or the essence of the poem, the main idea of the poem. Similarly, changes in the these mathematical expressions, it could lead to alterations in the idea or concept being described.
Analyzing (\ref{eq:9}), note that certain modifications to the expressions are neutral, as they continue to describe the same concept from a different perspective. Consider the \textit{right-hand side} of the equation.
\[
  16 + (2 + 2) = 16 + (4),\quad 16 + (4) = 16 + 4,\quad 16 + 4 = 20
\]
After this analysis, we may derive a new \textbf{\textit{equation}}.
\begin{equation} \label{eq:10}
  2 \cdot x + 10 = 20
\end{equation}
As we manipulate one expression, we can also manipulate both the \textit{left-hand side} and the \textit{right-hand side} of the \textbf{\textit{equation}} at the same time. There is no restriction preventing us from doing so. However, to maintain the truthfulness of the equation, we must apply the same changes to both expressions. Otherwise it would result in a false equation. Let us continue the analysis of (\ref{eq:10}).
\begin{equation} \label{eq:11}
  \begin{gathered}
    2 \cdot x + 10 = 20\\
    2 \cdot x + 10 + (-10) = (10 + 10) + (-10)\\
    2 \cdot x + 0 = 10\\
    2 \cdot x \cdot (\frac{1}{2}) = 10 \cdot (\frac{1}{2})\\
    x \cdot \frac{2}{2} = \frac{10}{2}\\
    x = 5
  \end{gathered}
\end{equation}
By manipulating the \textbf{\textit{equation}}, we were able to isolate $x$ from the original expression (\ref{eq:9}) and ascertain its true value. This clarification elucidates the nature of $x$, thereby transforming it from an \textit{unknown variable} into a \textit{known quantity} within the context of \textit{equation} (\ref{eq:9}).
This assertion holds true for $x = 5$, and one method to proof it is by substituting the value of $x$ into the expression of (\ref{eq:9}).
\[
  2 \cdot (5) + 10 = 10 + 10,\quad 10 + 10 = 20
\]

Furthermore, in (\ref{eq:11}), we introduce a neutral change $20 = (10 + 10)$. Such alterations are particularly significant, as they show the imaginative capacity of mathematicians to conceptualize and implement changes in expressions. Intuition plays a crucial role in navigating through these types of modifications.

I hope that this trivial example has demonstrated to you the importance of \textbf{\textit{equations}}. Ultimately, an \textit{equation} serves as a tool for analyzing mathematical expressions. But why is this necessary? Well, there are certain expressions, such as \textit{$\pi$} (\ref{eq:12}), whose true nature may not be immediately apparent. In such cases, we need to dissect the expressions of that object, compare them, and perhaps with enough ingenuity, we can formulate new \textbf{\textit{equations}} that reveal unknown properties of that object. In doing so, we create knowledge and advance in science.

As an expresssion to analyze could be the \textit{harmonic series}
\footnote{The name of the harmonic series derives from the concept of overtones or harmonics in music: the wavelengths of the overtones of a vibrating string are 
$\frac{1}{2}$, 
$\frac{1}{3}$, 
$\frac{1}{4}$, etc., of the string's fundamental wavelengths}, this is an infinite series of sums.

\[
  1 + \frac{1}{2} + \frac{1}{3} + \frac{1}{4} + \cdots
\]
In this series, all terms are positive unit fractions, and as the series progresses, each term becomes increasingly closer to zero. This leads to the hypothesis that at some point, the series will converge to a specific value. The mathematicians studying the various expressions within different theories arrive at this \textbf{\textit{equation}}.
\[
  1 + \frac{1}{2} + \frac{1}{3} + \frac{1}{4} + \cdots = \infty
\]
Returning with another example of an \textbf{\textit{equation}} that may initially appear false, how can we determine whether an equation is likely false? There isn't a specific method; this is accomplished through intuition and ingenuity.
\begin{equation} \label{eq:13}
  \text{For what $x$ number or numbers}\quad | \quad x + 1 = x + 2
\end{equation}
By subtracting $x$ from both sides, we arrive at a clearly false equation.
\[
\begin{gathered}
  x + 1 + (-x) = x + 2 + (-x)\\
  1 = 2
\end{gathered}
\]
We haven't discussed the implications of a mathematical expression or statement being false. Well, sometimes it's not straightforward, but in this case, it means that since $1 = 2$ is false, then $x + 1 = x + 2$ is also false. This implies that there are no solutions for $x$ in (\ref{eq:13}); in other words, there does not exist any number or numbers for which the expressions $x + 1$ and $x + 2$ are equal.
At first glance, it might seem that such expressions or equations lead to nothing. However, on the contrary, they do lead to something. Since it is evident that $1 = 2$ is false, because \textit{one} is \textit{less than} ($<$) \textit{two}, in other words:
\begin{equation} \label{eq:14}
  1 < 2
\end{equation}

Then it must be clear that by adding any number $x$ to both of these two expressions, $1$ and $2$, the expression (\ref{eq:14}) must hold true, arriving a new equation.
\begin{equation} \label{eq:15}
  \forall x \text{ number } \quad |\quad x + 1 < x + 2
\end{equation}
With this, we introduce the concept of \textit{\textbf{inequalities}}. These expressions, like equations, aid us in exploring and gaining a deeper understanding of mathematical concepts. There are a total of five types of \textit{inequalities}:
\begin{center}
  \begin{varwidth}{\textwidth}    
    \begin{enumerate}[label=(\Roman*)]
    \item \textit{less than}, $<$
    \item \textit{greater than}, $>$
    \item \textit{not equals}, $\neq$
    \item \textit{lesser than or equal to}, $\leq$
    \item \textit{greater than or equal to}, $\geq$
    \end{enumerate}
  \end{varwidth}
\end{center}
As an example is clear that for the number \textit{pi}.
\[
  \pi < 4 \text{ and } \pi > 3
\]
Because $\pi = 3.1415\ldots$, the use of \textit{\textbf{inequalities}} becomes important when mathematical expressions are difficult to define precisely. Inequalities help us grasp and clarify the range of values that a mathematical expression might represent. Additionally, some mathematical expressions are true only for certain values but not for all numbers. In these cases, using \textit{\textbf{inequalities}} allows us to narrow down the possible values of the variables. Consider the following example:
\[
  \forall x \neq 0,\quad \exists y \text{ number}\quad |\quad \frac{1}{x} = y
\]
Now, this expression would be false if we remove the part where $x \neq 0$, because in that case, it would imply something important: we would be saying that there exists a number such that $\frac{1}{0}$ is defined. At first glance, we might think that this expression exists, but consider this: what number multiplied by $0$ equals $1$? There are mathematicians who suggest that this expression is equal to $\infty$.
\[
  \frac{1}{0} = \infty
\]
This highlights the importance of paying attention to the numbers used within our expressions. It is crucial to delineate the possible values when necessary to ensure that we are not making incorrect assumptions or computations. So let's analyze another example of an \textit{\textbf{inequality}}:
\begin{equation} \label{eq:16}
  \text{For what $x$ number or numbers }\quad |\quad  4 \cdot x + 3 \geq 2
\end{equation}
Here, we can do the same thing as we do with \textit{equations}. Both parts of the \textit{\textbf{inequality}} can be manipulated (the \textit{left-hand side} and the \textit{right-hand side}), but to maintain the truth of the \textit{\textbf{inequality}}, we need to apply the same changes to both expressions. Observe how this is done:
\[
  \begin{gathered}
    4 \cdot x + 3 \geq 2\\
    4 \cdot x + 3 + (-3) \geq 2 + (-3)\\
    4 \cdot x + 0 \geq (-1)\\
    4 \cdot x \geq -1\\
    4 \cdot x \cdot (\frac{1}{4}) \geq -1 \cdot (\frac{1}{4})\\
    x \cdot (\frac{4}{4}) \geq \frac{(-1)}{4}\\
    x \geq - \frac{1}{4}\\
  \end{gathered}
\]
By manipulating this \textit{\textbf{inequality}}, we were able to isolate $x$ from the expression (\ref{eq:16}) and reveal its true nature within the context of the \textit{\textbf{inequality}}. This manipulation demonstrates that the expression holds true for $x \geq - \frac{1}{4}$. In other words:
\[
  \forall x \geq - \frac{1}{4} \quad | \quad 4 \cdot x + 3 \geq 2
\]

As with \textit{equations}, we can proof that this \textit{\textbf{inequality}} is true by substituting the value of $x$ into the expression (\ref{eq:16}) with \textit{any} number \textit{gretar than or equal to} $- \frac{1}{4}$.
\[
  4 \cdot (- \frac{1}{4}) + 3 \geq 2,\quad (- \frac{4}{4}) + 3 \geq 2, \quad (- 1) + 3 \geq 2,\quad 2 \geq 2
\]

\[
  4 \cdot (1) + 3 \geq 2,\quad 4 + 3 \geq 2, \quad 7 \geq 2
\]

Finally, the last concept to be discussed in this chapter is \textit{\textbf{identities}}. We have touched on the implications of the veracity of an \textit{equation} and \textit{inequality}, particularly in cases where it is false. Essentially, when an equation is true, it means that it exists, implying a connection between the mathematical expression and reality, suggesting that it can potentially be found in reality as an abstract representation. There are \textit{equations} with general expressions that aim to represent the form of a general type of a number or mathematical objects. We have already briefly discussed these types of abstract expressions, with the role of \textit{constants}. For example consider the next \textit{\textbf{identity}}
\begin{equation} \label{eq:17}
  c \cdot (a + b) = c \cdot a + c \cdot b
\end{equation}
Notice that I'm using $a, b, c$, which represent any fixed known constant or number. In this \textit{equation}, we are showing the distributive property (\ref{eq:17}). Generally speaking, an \textit{\textbf{identity}} is an equation that maintains its truth for any set of numbers. The purpose of using \textit{\textbf{identities}} is to express general properties of numbers, rather than specific instances with a limited set of numbers. Another common \textit{\textbf{identity}} is:
\begin{equation} \label{eq:18}
  (a + b)^2 = a^2 + 2ab + b^2
\end{equation}
Most of these \textit{equations} or \textit{\textbf{identities}} are utilized as formulas for problem-solving or computational purposes. The most simple example is:
\[
  a = a
\]
Which is clearly and undeniably true. There are numerous \textit{\textbf{identities}} in mathematics, for example an arithmetic \textit{\textbf{identity}}:
\begin{equation} \label{eq:19}
  1 + 2 + 3 + \cdots + n = \frac{n \cdot (n + 1)}{2}
\end{equation}

\section{\S Recap}
In the upcoming chapters, we will prove most of these \textit{\textbf{identities}}: (\ref{eq:17}), (\ref{eq:18}), and (\ref{eq:19}). As a conclusion to this chapter, I encourage the reader to dedicate time to consistently read and understand these texts. It's also beneficial to watch videos or read from more experienced and professional sources to gain deeper insights. While there may be some mistakes in my explanations, I am challenging myself to ensure my understanding of these topics. Throughout this chapter, discussing simple math topics, we have already demonstrated some of the characteristics mentioned in the introductory chapter. If you pay attention, we have already demonstrated logic by proving a few things, intuition in exploring these expressions, analysis to contemplate the meanings of these expressions, and both generality and individuality as we have discovered these properties in different objects. In the next chapter, we will utilize these tools to approach a clearer conceptualization of what numbers are and prove why they possess these properties.

And finally, I'd like to conclude with a simple reflection and an example. Mathematics is not about performing calculations quickly with complex and memorized methods. Mathematics is indeed related to philosophy, with the aim of understanding why the world behaves as it does. To be precise, all branches of mathematics are constructed upon \textit{definitions} and \textit{axioms} or \textit{postulates}. These \textit{axioms} or \textit{postulates} are propositions that are either logically self-evident and require no proof, or they may be propositions that are too challenging to prove as either true or false. In such cases, we accept them as assumptions and utilize them accordingly. \textit{Definitions}, on the other hand, serve as concise descriptions of the existence of objects that we encounter in our mathematical examinations. By examinations, I mean the process of exploring these \textit{definitions} and \textit{axioms} in the form of propositions as conjectures or hypotheses. Subsequently, through logical patterns, we seek to prove the validity of our propositions. This is essentially the essence of mathematics: understanding and proving the truth of statements. Therefore, being good at mental calculations alone does not necessarily mean being good at mathematics; while it may signify being good at calculations, it is only a fraction of what it means to be an excellent mathematician.

The first \textbf{\textit{axiom}} I would like to discuss in these notes is one of the most basic axioms encountered in geometry and algebra, which is expressed as follows:
\textit{If two objects are equal to a third, then they are equal to each other.}
Here, we have a conditional statement which ultimately implies something if it proves to be true. In mathematics, implications are denoted by $\implies$. Therefore, employing the mathematical symbolism introduced earlier, we can express this axiom as:
\begin{equation} \label{eq:20}
  \forall a, b, c \text{ numbers }\quad |\quad a = c\quad  \text{ and }\quad  b = c\quad \implies \quad a = b = c
\end{equation}
What we have here may seem trivial, yet it is essential. Without this statement, our substitutions would lack a foundational basis.








